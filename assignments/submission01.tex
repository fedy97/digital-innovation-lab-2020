\providecommand{\atd}{..}
\documentclass[../main.tex]{subfiles}

\begin{document}
    \chapter{Topic}\label{ch:topic}
    \section{Context}\label{sec:context}
    Nowadays a lot of data about different events are collected from multiple sources.
    For example the location through phones and cars’ GPS systems,
    the number of vehicles with their license plates through safety cameras,
    or the number of people moving in a city using the public transport service.
    Furthermore, Italy has started creating a national register of each citizen.
    Some data are already available to authorities,
    others can be made available through custom deals or bureaucratic procedures.


    \section{Focus}\label{sec:focus}
    The objective of our idea is to take advantage of those data sources in order to be always aware about what is happening in a city with a new level of detail and thus being able to adopt a new approach to security and safety monitoring. By extracting information from the gathered data, it will be possible to obtain a real-time monitoring system as well as a set of behavioural analysis based on the stored historical information. This will open a wide set of new approaches to security and safety monitoring, given the amount of information made available through the system in real-time.


    \section{Method}\label{sec:method}
    The core method idea is to apply the concept of a data warehouse architecture and its ETL components in order to obtain a proper data model and infrastructure, which gathers and organizes data within different sources.
    Once the data model is defined, it’s possible to implement multiple types of analysis on it.
    The target users will examine the information extracted from the data model using an intuitive web app composed of reports, graphs and interactive maps rendered according to pre-set filters or custom settings.


    \section{Benefit for target users}\label{sec:benefit-for-target-users}
    The system will be used by authorities, which have the duty of guaranteeing the public security of a city and probably there will be a task-force working in an operative room.
    The main benefit is identified in the enormous quantity of different useful information made available in a single system, and the possibility of implementing different monitoring logic (such as warnings in case of overcrowded areas of the city, strange behavior noticed by algorithms running on data or forecast based on historical stored information). This enables
    the authorities to act in a preventive and more coordinated way against threat to public safety.

\end{document}